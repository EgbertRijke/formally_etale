% latexmk -pdflatex='pdflatex %O %S' -pvc -pdf modal_descent.tex
\documentclass[9pt,twosided]{amsart}

\usepackage{hott}
\newcommand{\shape}{\int}
\newcommand{\drawpb}[1]{\arrow[#1, phantom, "\text{\scriptsize(pb)}" description]}
\newcommand{\CC}{\mathbb C}
\newcommand{\PP}{\mathbb P}

\title{Modal descent}
\author{Discussion with Shulman, Wellen, Frey, Awodey, Rijke}
\date{\today}

\begin{document}

\maketitle

Mike's observations:
\begin{enumerate}
\item Every factorization system $(\mathcal{E},\mathcal{M})$ has an underlying reflective
subuniverse, for which the modal types are those such that $X\to \unit$ is
in $\mathcal{M}$.  It is not clear whether every reflective subuniverse underlies
a factorization system.  One natural definition takes $\mathcal{E}$ to be the maps
inverted by the reflector $\modal$; another takes $\mathcal{M}$ to be the maps whose
$\modal$-naturality square is a pullback.  These two classes of maps are
always orthogonal, but factorizations do not always exist; i.e. they
are a "prefactorization system".  The prefactorization systems
obtained in this way from a reflective subuniverse are called
*reflective*; they are precisely those for which $\mathcal{E}$ satisfies
2-out-of-3.
\item 
\begin{enumerate}
\item Every modality is, in particular, a reflective subuniverse, so the
foregoing applies to it.  In this case, however, the putative explicit
definitions of $\mathcal{E}$ and $\mathcal{M}$ are always a factorization system, with
factorizations constructed by pulling back in the $\modal$-naturality square;
thus we always obtain a reflective factorization system.  The
modalities are precisely the reflective subuniverses that have \emph{stable
units}, i.e. all the modal units $X\to\modal X$ are $\modal$-connected.  In \cite{CassidyHebertKelly}
there are two weaker conditions on a reflective subcategory that
ensure that the reflective prefactorization system has factorizations:
being \emph{simple} (equivalent to the fact that factorizations can be
constructed by pulling back in the $\modal$-naturality square) and being
\emph{semi-left-exact}.  It is not clear whether in HoTT there exist
semi-left-exact reflective subuniverses that are not modalities, or
simple ones that are not semi-left-exact, or even reflective
subuniverses that are not simple.
\item Every modality also gives rise to a different factorization
system, in which the right class is the maps with $\modal$-modal fibers and
the left class is the maps with $\modal$-connected fibers.  This
factorization system is \emph{stable}, i.e. the left class (and hence the
factorizations) are stable under pullback.  Moreover, every stable
factorization system arises in this way from a unique modality.
\end{enumerate}
\item The above two factorization systems associated to a modality do not
in general coincide.  They can coincide only if the $\modal$-connected maps
satisfy 2-out-of-3 (since the $\modal$-inverted maps always do), and only if
the $\modal$-inverted maps are stable under pullback (since the $\modal$-connected
maps always are).  However, both of these properties are sufficient
to imply that $\modal$ is fully left exact, which in turn implies that the
two factorization systems coincide.
\item Another way of extending a reflective subuniverse to a
factorization system is that if it's accessible, we can choose a
generating family and then use the same generating family to generate
a factorization system.  This works for any accessible reflective
subuniverse, but in principle, at least, it seems as if the result may
depend on the generating family chosen, and I don't see any reason for
it to coincide with the reflective (pre)factorization system.  I also
don't know whether anything special happens for modalites.
\item If we have a factorization system $(\mathcal{E},\mathcal{M})$,
with underlying reflective subuniverse $\modal$, then all maps between
$\modal$-modal types are in $\mathcal{M}$, hence so are any of their pullbacks.  Thus,
the reflective factorization system, if it exists, is the
factorization system with the smallest class $\mathcal{M}$ that has a given
underlying reflective subuniverse.  On the other hand, again for any
$(\mathcal{E},\mathcal{M})$, any pullback of an $\mathcal{M}$-map is again in $\mathcal{M}$, hence all fibers of an
$\mathcal{M}$-map are $\modal$-modal; thus the stable factorization system, if it exists
(i.e. if $\modal$ is a modality), is the factorization system with the
\emph{largest} class $\mathcal{M}$ that has a given underlying reflective subuniverse.
\end{enumerate}

In a presheaf topos, a natural transformation is cartesian if and only if it is right orthogonal to all maps between representables. 
Question: What do we get if we localize inside an
arbitrary presheaf topos at the maps between representables?  A
general presheaf topos is not cohesive, so the discrete objects are
not fully embedded and we can't produce a modality by "taking the
colimit and then the discrete object".  But this localization seems to
be trying to give us the closest possible approximation to such.  As
an extreme example in the other direction, if the domain of presheaves
is a discrete category, there are no nontrivial maps between
representables, so every transformation is cartesian and the
localization is the identity functor.

\section{Preliminaries}
\begin{enumerate}
\item A commuting square
\begin{equation*}
\begin{tikzcd}
A \arrow[d,swap,"f"] \arrow[r] & B \arrow[d,"g"] \\
X \arrow[r,swap,"h"] & Y
\end{tikzcd}
\end{equation*}
is a pullback square if and only if the induced map on fibers
\begin{equation*}
\prd{x:X}\fib{f}{x}\to\fib{g}{h(x)}
\end{equation*}
is a fiberwise equivalence.
\item For any modality $\modal$ the following are equivalent:
\begin{enumerate}
\item All propositions are modal.
\item All $\modal$-connected maps are surjective.
\end{enumerate}
In particular it follows that all $n$-connected maps are surjective, for $n\geq -1$.
\end{enumerate}

\section{\texorpdfstring{$\modal$}{○}-\'etale maps}

\begin{defn}
We say that a map $f:A\to B$ is \define{$\modal$-\'etale} if the square
\begin{equation*}
\begin{tikzcd}
A \arrow[r,"f"] \arrow[d,swap,"\eta"] & B \arrow[d,"\eta"] \\
\modal A \arrow[r,swap,"\modal f"] & \modal B
\end{tikzcd}
\end{equation*}
is a pullback square. We will write
\begin{equation*}
\isetale(f)\defeq\ispullback(f,\eta_A,\natunit_\modal(f)).
\end{equation*}
\end{defn}

It is immediate from the definition that any equivalence is $\modal$-\'etale, and that the $\modal$-\'etale maps are closed under composition.

\begin{eg}\label{eg:etale_prop}
We claim that a map $f:A\to B$ is $\brck{\blank}$-\'etale if and only if $A\to \isequiv(f)$. Examples of maps that satisfy this condition include equivalences, maps between propositions, and any map of the form $\emptyt\to B$.

To see that if $f:A\to B$ is $\modal$-\'etale, then $A\to\isequiv(f)$, consider the pullback square
\begin{equation*}
\begin{tikzcd}
A \arrow[r] \arrow[d,swap,"f"] & \brck{A} \arrow[d,"\brck{f}"] \\
B \arrow[r] & \brck{B},
\end{tikzcd}
\end{equation*}
and let $a:A$. Then both $\brck{A}$ and $\brck{B}$ are contractible, so $\brck{f}:\brck{A}\to\brck{B}$ is an equivalence. Since equivalences are stable under pullback it follows that $f$ is an equivalence.

Now suppose that $A\to \isequiv(f)$. Since $\isequiv(f)$ is a proposition, we also have $\brck{A}\to\isequiv(f)$. To see that the gap map
\begin{equation*}
A \to B\times_{\brck{B}}\brck{A}
\end{equation*}
is an equivalence, we will show that its fibers are contractible. Let $b:B$, $x:\brck{A}$ and $p:\bproj{b}=\brck{f}(x)$. Since $\brck{A}\to\isequiv(f)$, it follows that $f$ is an equivalence. Then $\brck{f}$ is also an equivalence, from which it follows that the naturality square is a pullback square. We conclude that the fibers of the gap map are contractible. 
\end{eg}

\begin{lem}\label{lem:etale_modal}
Any map between $\modal$-modal types is formaly \'etale.
\end{lem}

\begin{proof}
Suppose $f:X\to Y$ is a map between $\modal$-modal types. Then the top and bottom maps in the square
\begin{equation*}
\begin{tikzcd}
X \arrow[r] \arrow[d] & \modal X \arrow[d] \\
Y \arrow[r] & \modal Y
\end{tikzcd}
\end{equation*}
are equivalences. Therefore this square is a pullback square, so $f$ is $\modal$-\'etale.
\end{proof}

In this section our goal is to characterize the $\modal$-\'etale maps for the $n$-truncations. We call such maps \define{$n$-\'etale}. Note that the case of propositional truncation has already been addressed in \cref{eg:etale_prop}.

\begin{lem}\label{lem:etale_char}
Let $\modal$ be a modality for which all propositions are modal, and consider a map $f:A\to B$. The following are equivalent:
\begin{enumerate}
\item $f$ is $\modal$-\'etale.
\item The commuting square
\begin{equation*}
\begin{tikzcd}[column sep=large]
A\times_{\modal A} A \arrow[d,swap,"\pi_1"] \arrow[r,"{f\times_{\modal f} f}"] & B\times_{\modal B} B \arrow[d,"\pi_1"] \\
A \arrow[r,swap,"f"] & B
\end{tikzcd}
\end{equation*}
is a pullback square.
\end{enumerate}
\end{lem}

\begin{rmk}
In the special case of $(-1)$-truncation, the characterization of \cref{lem:etale_char} asserts that a map $f:A\to B$ is $(-1)$-\'etale if and only if the square
\begin{equation*}
\begin{tikzcd}
A\times A \arrow[d,swap,"\pi_1"] \arrow[r,"{f\times f}"] & B\times B \arrow[d,"\pi_1"] \\
A \arrow[r,swap,"f"] & B
\end{tikzcd}
\end{equation*}
is a pullback square.
\end{rmk}

\begin{proof}
Suppose first that $f$ is $\modal$-\'etale, and consider the commuting cube
\begin{equation*}
\begin{tikzcd}
& A\times_{\modal A} A \arrow[dl] \arrow[d] \arrow[dr] \\
A \arrow[d] & B\times_{\modal B} B \arrow[dl] \arrow[dr] & A \arrow[dl,crossing over] \arrow[d] \\
B \arrow[dr] & \modal A \arrow[from=ul,crossing over] \arrow[d] & B \arrow[dl] \\
& \modal B
\end{tikzcd}
\end{equation*}
Since the top, bottom, and both front squares are pullback squares, it follows that both back squares are pullback. This proves that (i) implies (ii).

Now suppose that (ii) holds. Then the map
\begin{equation*}
\fib{\modalunit}{\modalunit(a)}\to \fib{\modalunit}{\modalunit(f(a))}
\end{equation*}
is an equivalence for every $a:A$. Since all propositions are assumed to be modal, it follows that
\begin{equation*}
\fib{\modalunit}{t}\to \fib{\modalunit}{\modal f(t)}
\end{equation*}
is an equivalence for every $t:\modal A$. Thus it follows that the square
\begin{equation*}
\begin{tikzcd}
A \arrow[d,swap,"\modalunit"] \arrow[r] & B \arrow[d,"\modalunit"] \\
\modal A \arrow[r] & \modal B
\end{tikzcd}
\end{equation*}
is a pullback square.
\end{proof}

\begin{cor}
If $f:A\to B$ is $\modal$-\'etale, then the square
\begin{equation*}
\begin{tikzcd}[column sep=large]
A \arrow[d,swap,"\delta_{\modalunit}"] \arrow[r,"f"] & B \arrow[d,"\delta_{\modalunit}"] \\
A\times_{\modal A} A \arrow[r,swap,"f\times_{\modal f}f"] & B\times_{\modal B} B
\end{tikzcd}
\end{equation*}
is a pullback square.
\end{cor}

\begin{proof}
Consider the diagram
\begin{equation*}
\begin{tikzcd}[column sep=large]
A \arrow[d,swap,"\delta_{\modalunit}"] \arrow[r,"f"] & B \arrow[d,"\delta_{\modalunit}"] \\
A\times_{\modal A} A \arrow[d,swap,"\pi_1"] \arrow[r,"{f\times_{\modal f} f}"] & B\times_{\modal B} B \arrow[d,"\pi_1"] \\
A \arrow[r,"f"] & B
\end{tikzcd}
\end{equation*}
The bottom square is a pullback square by \cref{lem:etale_char}, and the outer rectangle is a pullback since both vertical composites are homotopic to the respective identity functions. Therefore the top square is a pullback.
\end{proof}

\begin{thm}
A map $f:A\to B$ is $0$-\'etale if and only if for each $a:A$ the restriction
\begin{equation*}
\begin{tikzcd}
\sm{x:A}\brck{a=x} \arrow[d,swap,"\proj 1"] \arrow[r,densely dotted,"f"] & \sm{y:B}\brck{f(a)=y} \arrow[d,"\proj 1"] \\
A \arrow[r,swap,"f"] & B
\end{tikzcd}
\end{equation*}
of $f$ to the connected component at $a$ of $A$ is an equivalence. 
\end{thm}

\begin{proof}
By \cref{lem:etale_char} and the fact that $\eqv{(\tproj{0}{a}=\tproj{0}{x})}{\brck{a=x}}$, it follows that $f$ is $0$-\'etale if and only if the square
\begin{equation*}
\begin{tikzcd}[column sep=huge]
\sm{a,x:A}\brck{a=x} \arrow[d,swap,"\pi_1"] \arrow[r,"\total{\brck{\apfunc{f}}}"] & \sm{b,y:B}\brck{b=y} \arrow[d,"\pi_1"] \\
A \arrow[r,swap,"f"] & B
\end{tikzcd}
\end{equation*}
is a pullback square. Furhtermore, this square is a pullback if and only if the induced map
\begin{equation*}
\Big(\sm{x:A}\brck{a=x}\Big)\to\Big(\sm{y:B}\brck{f(a)=y}\Big)
\end{equation*}
is an equivalence, for each $a:A$.
\end{proof}

\section{Modal descent}

The following theorem can be seen as a `modal flattening lemma'.
\begin{thm}\label{thm:etale_flattening}
Consider a pullback square
\begin{equation*}
\begin{tikzcd}
E' \arrow[d,swap,"{p'}"] \arrow[r,"g"] & E \arrow[d,"p"] \\
B' \arrow[r,swap,"f"] & B
\end{tikzcd}
\end{equation*}
with $H:f\circ p'\htpy p\circ g$, where $X$ and $Y$ are modal types. Then the square
\begin{equation*}
\begin{tikzcd}
\modal E' \arrow[r,"\tilde{g}"] \arrow[d,swap,"{\modal p'}"] & E \arrow[d,"p"] \\
\modal B \arrow[r,swap,"\tilde{f}"] & B
\end{tikzcd}
\end{equation*}
is a pullback square, where $\tilde{f}$ and $\tilde{g}$ are the unique extensions of $f$ and $g$ along the modal units of $B'$ and $E'$, respectively.
\end{thm}

\begin{proof}
Consider the diagram
\begin{equation*}
\begin{tikzcd}
E' \arrow[r,"{\mathsf{gap}(p',g,H)}"] \arrow[d,swap,"{p'}"] &[2.5em] \modal B'\times_{B} E \arrow[r,"\pi_2"] \arrow[d,swap,"\pi_1"] & E \arrow[d,"p"] \\
B' \arrow[r,swap,"\modalunit"] & \modal B' \arrow[r,swap,"\tilde{f}"] & B
\end{tikzcd}
\end{equation*}
In this diagram, the square on the right is a pullback by definition, and the outer rectangle is a pullback by assumption, so the square on the left is also a pullback. Therefore the gap map $E'\to \modal B'\times_B E$ is $\modal$-connected. Moreover, since the modal types are closed under pullbacks it follows that $\modal B'\times_B E$ is modal, and therefore it follows that $\pi_2:\modal B'\times_B E\to E$ is a modal map. Therefore the composite
\begin{equation*}
\begin{tikzcd}
E' \arrow[r,"{\mathsf{gap}(p',g,H)}"] &[2.5em] \modal B'\times_{B} E \arrow[r,"\pi_2"] & E 
\end{tikzcd}
\end{equation*}
factors $g$ as a $\modal$-connected map followed by a $\modal$-modal map. Of course, another such factorization is the composite $g\htpy \tilde{g}\circ\modalunit$. Since factorizations are unique, the claim follows.
\end{proof}

Using modal flattening we establish partial left exactness of the modality.

\begin{cor}\label{cor:etale_lex}
Consider a pullback square
\begin{equation*}
\begin{tikzcd}
A' \arrow[d,swap,"{f'}"] \arrow[r] & A \arrow[d,"f"] \\
B' \arrow[r] & B,
\end{tikzcd}
\end{equation*}
where $f$ is assumed to be $\modal$-\'etale. Then the square
\begin{equation*}
\begin{tikzcd}
\modal A' \arrow[d,swap,"{\modal f'}"] \arrow[r] & \modal A \arrow[d,"\modal f"] \\
\modal B' \arrow[r] & \modal B,
\end{tikzcd}
\end{equation*}
is again a pullback square.
\end{cor}

\begin{proof}
Since $f$ is assumed to be $\modal$-\'etale, the square on the right in the diagram
\begin{equation*}
\begin{tikzcd}
A' \arrow[r] \arrow[d,swap,"{f'}"] & A \arrow[r] \arrow[d,swap,"f"] & \modal A \arrow[d,"\modal f"] \\
B' \arrow[r] & B \arrow[r] & \modal B
\end{tikzcd}
\end{equation*}
is a pullback square. Therefore the outer rectangle is a pullback square by the pullback pasting lemma. Now the claim follows from modal flattening \cref{thm:etale_flattening}, using the outer rectangle.
\end{proof}

\begin{cor}\label{cor:etale_pb}
Consider a pullback square
\begin{equation*}
\begin{tikzcd}
E' \arrow[d,swap,"{p'}"] \arrow[r,"g"] & E \arrow[d,"p"] \\
B' \arrow[r,swap,"f"] & B
\end{tikzcd}
\end{equation*} 
and suppose that $p:E\to B$ is $\modal$-\'etale. Then $p':E'\to B'$ is $\modal$-\'etale.
\end{cor}

\begin{proof}
Consider the commuting cube
\begin{equation*}
\begin{tikzcd}
&[-.5ex] E' \arrow[dr] \arrow[d] \arrow[dl] \\
\modal E' \arrow[d] & B' \arrow[dl] \arrow[dr] & E \arrow[dl,crossing over] \arrow[d] \\
\modal B' \arrow[dr] & \modal E \arrow[d] \arrow[from=ul,crossing over] & B \arrow[dl] \\
& \modal B.
\end{tikzcd}
\end{equation*}
The vertical squares on the back right and front right are pullback squares by assumption.
Then it follows from \cref{cor:etale_lex} that the vertical square on the front left is a pullback square.
Therefore the square on the back left is a pullback square by the pullback pasting property.
\end{proof}

\begin{defn}
Let $X$ be a type. We will define an operation
\begin{equation*}
\etmap:\Big(\sm{A:\UU_\modal}A\to\modal X\Big)\to\Big(\sm{Y:\UU}{g:Y\to X}\isetale(g)\Big)
\end{equation*}
\end{defn}

\begin{constr}
Given a map $f:A\to \modal X$ we take the pullback
\begin{equation*}
\begin{tikzcd}
X\times_{\modal X}A \arrow[d,swap,"\pi_1"] \arrow[r,"\pi_2"] & A \arrow[d,"f"] \\
X \arrow[r,swap,"\modalunit"] & \modal X.
\end{tikzcd}
\end{equation*}
Then the map $\pi_1:X\times_{\modal X}A\to X$ is $\modal$-\'etale by \cref{lem:etale_modal,cor:etale_pb}.
\end{constr}

The following is a descent theorem for $\modal$-\'etale maps.

\begin{thm}[Modal descent]\label{thm:modal_descent}
For any modality $\modal$, and any type $X$, the operation
\begin{equation*}
\etmap:\Big(\sm{A:\UU_\modal}A\to\modal X\Big)\to\Big(\sm{Y:\UU}{g:Y\to X}\isetale(g)\Big)
\end{equation*}
is an equivalence.
\end{thm}

\begin{proof}
If $g:Y\to X$ is $\modal$-\'etale, then the square
\begin{equation*}
\begin{tikzcd}
Y \arrow[d,swap,"g"] \arrow[r,"\modalunit"] & \modal Y \arrow[d,"\modal g"] \\
X \arrow[r,swap,"\modalunit"] & \modal X
\end{tikzcd}
\end{equation*}
is a pullback square. Therefore $g:Y\to X$ is in the fiber of $\etmap$ at $\modal g : \modal Y\to\modal X$. 

It remains to show that for any map $f:A\to\modal X$ with modal domain, there is an equivalence $\eqv{A}{\modal (X\times_{\modal X} A)}$ such that the triangle
\begin{equation*}
\begin{tikzcd}
A \arrow[dr,swap,"f"] \arrow[rr,"\eqvsym"] & & \modal (X\times_{\modal X} A) \arrow[dl,"\modal(\etmap(f))"] \\
& \modal X
\end{tikzcd}
\end{equation*}
commutes. To see this, note that both $f\circ \pi_2$ and $\modal(\etmap(f))\circ \modalunit$ factor the same map as a $\modal$-connected map followed by a modal map, so the claim follows from uniqueness of factorizations.
\end{proof}

\begin{cor}
Suppose $P:X\to\UU_\modal$ is a family of modal types such that the projection map $\proj 1:\big(\sm{x:X}P(x)\big)\to X$ is $\modal$-\'etale. Then $P$ has a unique extension
\begin{equation*}
\begin{tikzcd}
X \arrow[d,swap,"\modalunit"] \arrow[r,"P"] & \UU_\modal. \\
\modal X \arrow[ur,densely dotted,swap,"\tilde{P}"] 
\end{tikzcd}
\end{equation*}
It follows that the square commuting square
\begin{equation*}
\begin{tikzcd}
\sm{x:X}P(x) \arrow[d,swap,"\proj 1"] \arrow[r] & \sm{t:\modal X}\tilde{P}(t) \arrow[d,"\proj 1"] \\
X \arrow[r,swap,"\modalunit"] & \modal X
\end{tikzcd}
\end{equation*}
is a pullback square. In particular the top map is $\modal$-connected, so this square is in fact a $\modal$-naturality square.
\end{cor}

\begin{lem}
If $f$ is $n$-\'etale, then it is right orthogonal to $\unit\to\sphere{n+1}$.
\end{lem}

\begin{proof}
Suppose $f:X\to Y$ is $n$-\'etale, and consider the commuting cube
\begin{equation*}
\begin{tikzcd}
&[-1ex] X^{\sphere{n+1}} \arrow[dl] \arrow[d] \arrow[dr] &[2ex] \\
\trunc{n}{X}^{\sphere{n+1}} \arrow[d] & Y^{\sphere{n+1}} \arrow[dl] \arrow[dr] & X \arrow[dl,crossing over] \arrow[d] \\
\trunc{n}{Y}^{\sphere{n+1}} \arrow[dr] & \trunc{n}{X} \arrow[d] \arrow[from=ul,crossing over] & Y \arrow[dl] \\
& \trunc{n}{Y}
\end{tikzcd}
\end{equation*}
In this cube the front right square is a pullback square by the assumption that $f$ is $n$-\'etale. The back left square is an exponent of this pullback square, so it is again pullback. The front left square is a pullback square because its top and bottom map are both equivalences. Therefore we conclude that the back right square is a pullback square, which shows that $f$ is right orthogonal to the map $\unit\to\sphere{n+1}$.
\end{proof}

\begin{conj}
Let $f:X\to Y$ be a map, and let $n\geq -1$. The following are equivalent:
\begin{enumerate}
\item $f$ is $n$-\'etale.
\item $f$ is right orthogonal to the maps $\sphere{n+1}\to 1$ and $\unit\to\sphere{n+1}$. 
\end{enumerate}
\end{conj}

\begin{proof}[Incomplete proof]
By \cref{thm:modal_descent} it suffices to show that the type $\trunc{n}{X}\to\UU^{\leq n}$ of families of $n$-types over $\trunc{n}{X}$ is equivalent to the type of families $P:X\to\UU^{\leq n}$ of $n$-types over $X$, for which the square
\begin{equation*}
\begin{tikzcd}[column sep=huge]
\Big(\sm{x:X}P(x)\Big)^{\sphere{n+1}} \arrow[r,"{h\mapsto h(\base)}"] \arrow[d,swap,"\proj 1\circ\blank"] & \sm{x:X}P(x) \arrow[d,"\proj 1"] \\
X^{\sphere{n+1}} \arrow[r,swap,"{h\mapsto h(\base)}"] & X
\end{tikzcd}
\end{equation*}
is a pullback square. Note that this square is a pullback square if and only if the square
\begin{equation*}
\begin{tikzcd}[column sep=9em]
\sm{f:X^{\sphere{n+1}}}\prd{t:\sphere{n+1}}P(f(t)) \arrow[r,"{(f,g)\mapsto (f(\base),g(\base))}"] \arrow[d,swap,"\proj 1"] & \sm{x:X}P(x) \arrow[d,"\proj 1"] \\
X^{\sphere{n+1}} \arrow[r,swap,"{f\mapsto f(\base)}"] & X
\end{tikzcd}
\end{equation*}
is a pullback square. This is equivalent to showing that the map
\begin{equation*}
g\mapsto g(\base):\Big(\prd{t:\sphere{n+1}}P(f(t))\Big)\to P(f(\base))
\end{equation*}
is an equivalence for every $f:\sphere{n+1}\to X$. The fiber of this map at $p:P(f(\base))$ is equivalent to the type
\begin{equation*}
\sm{q:P(f(\base))}\prd{t:\sphere{n}}\mathsf{tr}_P(\glue(t)
\end{equation*}
....
\end{proof}

\section{The reflective factorization system}

In this section we investigate a second factorization system that can be obtained from any modality, of which the right class is the class of $\modal$-\'etale maps. The left class is the class of \emph{$\modal$-equivalences}.

\begin{defn}
We say that a map $f:A\to B$ is an \define{$\modal$-equivalence} if $\modal f:\modal A\to \modal B$ is an equivalence.
\end{defn}

\begin{rmk}
The difference between the notions of $\modal$-equivalences and $\modal$-connected maps is best explained by an example. In the case of $n$-truncation, the $n$-equivalences are precisely the maps that induce isomorphisms on the first $n$ homotopy groups. The $n$-connected maps are the maps that induce isomorphisms on the first $n$ homotopy groups, and moreover induce an epimorphism on the $(n+1)$-st homotopy group. 

We also note that the $n$-equivalences are not stable under pullbacks, whereas the $n$-connected maps are. Consider for instance the pullback square
\begin{equation*}
\begin{tikzcd}
\loopspace {\sphere{n+1}} \arrow[r] \arrow[d] & \unit \arrow[d] \\
\unit\arrow[r] & \sphere{n+1}
\end{tikzcd}
\end{equation*}
Here the map on the right is an $n$-equivalence, since $\sphere{n+1}$ is $n$-connected. However, the map on the left is not an $n$-equivalence, since the $n$-th homotpy group of $\loopspace{\sphere{n+1}}$ is not trivial: it is the $(n+1)$-st homotopy group of $\sphere{n+1}$, which is $\Z$.
\end{rmk}

\begin{defn}
The \define{reflective factorization system} associated to a modality $\modal$ consists of the $\modal$-equivalences as the left class, and the $\modal$-\'etale maps as the right class.
\end{defn}

Our goal in this section is to show that the reflective factorization system associated to a modality is an orthogonal factorization system.

\begin{lem}\label{lem:3for2_mequiv}
The $\modal$-equivalences satisfy the 3-for-2 property: given a commuting triangle
\begin{equation*}
\begin{tikzcd}
A \arrow[rr,"h"] \arrow[dr,swap,"f"] & & B \arrow[dl,"g"] \\
& C,
\end{tikzcd}
\end{equation*}
if any two of $f$, $g$, and $h$ are $\modal$-equivalences, then so is the third.
\end{lem}

\begin{proof}
Apply $\modal$ to the commuting triangle, and use the 3-for-2 property of equivalences.
\end{proof}

\begin{lem}\label{lem:modal_equivalence}
For a map $f : A \to B$ the following are equivalent:
\begin{enumerate}
\item $f$ is an $\modal$-equivalence.
\item For any modal type $X$, the precomposition map
\begin{equation*}
\precomp{f} : (B \to X) \to (A \to X)
\end{equation*}
is an equivalence.
\end{enumerate}
\end{lem}

\begin{proof} 
Suppose first that $f$ is an $\modal$-equivalence, and let $X$ be $\modal$-modal. Then the square
\begin{equation*}
\begin{tikzcd}
X^B \arrow[r,"\precomp{f}"] \arrow[d,swap,"\precomp{\eta}"] & X^A \arrow[d,"\precomp{\eta}"] \\
X^{\modal B} \arrow[r,swap,"\precomp{\modal f}"] & X^{\modal A}
\end{tikzcd}
\end{equation*}
commutes. In this square the two vertical maps are equivalences by the universal property of modalization, and the bottom map is an equivalence since $\modal f$ is an equivalence. Therefore the map $\precomp{f}:X^B\to X^A$ is an equivalence, as desired.

Conversely, assume that $\precomp{f} : X^B \to X^A$ is an equivalence for every $\modal$-modal type $X$. By the square above it follows that $\precomp{\modal f}:X^{\modal B}\to X^{\modal A}$ is an equivalence for every $\modal$-modal type $X$. The fiber of $\modal A^{\modal B}\to \modal A^{\modal A}$ at $\idfunc:\modal A\to \modal A$ is contractible, so we obtain a retraction $g$ of $\modal f$. To see that $g$ is also a section observe that the fiber of $\modal B^{\modal B}\to \modal B^{\modal A}$ at $\modal f$ is contractible. This fiber contains $(\idfunc[\modal B],\refl{\modal f})$. However, we also have an identification $p:\precomp{\modal f}(\modal f\circ g)=\modal f$, since
\begin{equation*}
\precomp{\modal f}(\modal f\circ g)\jdeq (\modal f \circ g)\circ \modal f\jdeq \modal f \circ (g\circ \modal f) = \modal f. 
\end{equation*}
Therefore $(\modal f\circ g,p)$ is in the fiber of $\precomp{\modal f}:\modal B^{\modal B}\to \modal B^{\modal A}$ at $\modal f$. By the contractibility of the fibers it follows that $(\modal f\circ g,p)=(\idfunc[\modal B],\refl{\modal f})$, so it follows that $\modal f\circ g=\idfunc[\modal B]$. In other words, $g$ is both a retraction and a section of $\modal f$, so $\modal f$ is an equivalence.
\end{proof}

\begin{cor}\label{cor:mequiv_mconn}
Every $\modal$-connected map is a $\modal$-equivalence.
\end{cor}

\begin{lem}\label{lem:rfs_factor}
Every map factors as a $\modal$-equivalence followed by a $\modal$-\'etale map.
\end{lem}

\begin{proof}
Consider a map $f:A\to B$, and the diagram
\begin{equation*}
\begin{tikzcd}
A \arrow[ddr,bend right=15,swap,"f"] \arrow[drr,bend left=15,"\modalunit"] \arrow[dr,"\mathsf{gap}" description] \\
& B\times_{\modal B} \modal A \arrow[d,swap,"\pi_1"] \arrow[r,"\pi_2"] & \modal A \arrow[d,"\modal f"] \\
& B \arrow[r,swap,"\modalunit"] & \modal B.
\end{tikzcd}
\end{equation*}
Then $\pi_1:B\times_{\modal B} \modal A\to B$ is a pullback of a map between modal types, so it is $\modal$-\'etale by \cref{cor:etale_pb}. Furthermore, the map $\pi_2:B\times_{\modal B}\modal A\to \modal A$ is a pullback of a $\modal$-connected map, so it is $\modal$-connected. It follows from \cref{cor:mequiv_mconn} that it is a $\modal$-equivalence. Since the modal unit $\modalunit :A\to\modal A$ is also $\modal$-connected, and therefore a $\modal$-equivalence, we obtain by the 3-for-2 property of $\modal$-equivalences established in \cref{lem:3for2_mequiv} that the gap map is also a $\modal$-equivalence.
\end{proof}

Recall that a map $i:A\to B$ is said to be \define{left orthogonal} to a map $f:X\to Y$, or that the map $f$ is \define{right orthogonal} to the map $i$,
if each commuting square of the form
\begin{equation*}
\begin{tikzcd}
A \arrow[r] \arrow[d,swap,"i"] & X \arrow[d,"f"] \\
B \arrow[r] & Y
\end{tikzcd}
\end{equation*}
has a unique diagonal filler. Following \cite{AnelBiedermanFinsterJoyal}, we note that this can be expressed conviniently as the condition that the square
\begin{equation*}
\begin{tikzcd}
X^B \arrow[r,"f\circ\blank"] \arrow[d,swap,"\blank\circ i"] & Y^B \arrow[d,"\blank\circ i"] \\
X^A \arrow[r,swap,"f\circ\blank"] & Y^A
\end{tikzcd}
\end{equation*}
is a pullback square.

\begin{lem}\label{lem:rfs_orthogonal}
The class of $\modal$-equivalences is left orthogonal to the class of $\modal$-\'etale maps.
\end{lem}

\begin{proof}
We have to show that for every $\modal$-equivalence $i:A\to B$, and every $\modal$-\'etale map $f:X\to Y$, the square
\begin{equation*}
\begin{tikzcd}
X^B \arrow[r] \arrow[d] & Y^B \arrow[d] \\
X^A \arrow[r] & Y^A
\end{tikzcd}
\end{equation*}
is a pullback square. Consider the commuting cube
\begin{equation*}
\begin{tikzcd}
&[-1ex] X^B \arrow[dl] \arrow[d] \arrow[dr] \\
(\modal X)^B \arrow[d] & X^A \arrow[dl] \arrow[dr] & Y^B \arrow[d] \arrow[dl,crossing over] \\
(\modal X)^A \arrow[dr] & (\modal Y)^B \arrow[from=ul,crossing over] \arrow[d] & Y^A \arrow[dl] \\
& (\modal Y)^A
\end{tikzcd}
\end{equation*}
In this cube the top and bottom squares are pullback by the assumption that $f$ is $\modal$-\'etale and the fact that exponents of pullback squares are again pullback squares. Furthermore, the square in the front left is pullback, because the two vertical maps are equivalences by the assumption that $i:A\to B$ is a $\modal$-equivalence. Therefore we conclude that the square in the back right is also a pullback square, as desired.
\end{proof}

\begin{cor}
For any map $f:X\to Y$, the type of factorizations into a $\modal$-connected map followed by a $\modal$-\'etale map is contractible.
\end{cor}

\section{Applications}

So far, we discussed the n-truncations as an example from plain Homotopy Type Theory.
Viewing Homotopy Type Theory as an internal language of ($\infty$,1)-toposes, 
which is crucial for the following,
special features of a particular topos may be introduced into the type theory as a modality.
Important parts of the coverings theory for topological spaces hold for a general abstract modality.
In the following, this will be explained for a special type theory which admits recovery of the classic topological situation.

In \cite{ShulmanRealCohesion} Mike Shulman introduces Real Cohesive Homotopy Type Theory, as a candidate for an internal language
of some of the ($\infty$,1)-toposes called \emph{cohesive}, a higher analog 
of Lawvere's axiomatic cohesion \cite{Lawvere07} developed by Urs Schreiber \cite{SchreiberDcct}.

In Real Cohesive Homotopy Type Theory,
which will just be called Real Cohesion in the following, 
some well behaved topological spaces, like, for example, topological manifolds
are supposed to be included in the theory.
It is important to note, that the types corresponding to these topological spaces
are 0-types in Real Cohesion.
This can lead to confusion with the common explanation for the Identity types in Homotopy Type Theory,
as paths in a space and care has to be taken to separate concept of equality
and topological paths, i.e. maps $\gamma:\RR\to X$ from the 0-type $\RR$ representing the real line with the euclidean topology.
Let $\sphere{n}$ denote the topological sphere and $S^n$ the higher inductive type 
introduced in \cite{UFP}.

For the present work, the shape modality ``$\shape$'' from Real Cohesion is of special interest.
It maps topological spaces to their \emph{homotopy type}, so for example $\shape \sphere{1} =S^1$ 
and $\shape \RR =1$. 
In a 1-topos cohesive over Set, the functor $\Delta\circ\pi_0$ maps a sheaf to the sheaf constantly its set of connected components.
The $\shape$ is a higher analog of this functor that extracts homotopical information on all h-levels, not just level 0.
So if $X$ represents a topological space with a point $\ast: X$, 
then $\shape X$ is also pointed and the n-th homotopy group of $X$ as a topological space 
could be retrieved from its shape 
as $\pi_n(X):\equiv\trunc{0}{\loopspace^n\shape X}$.

Like $\Delta\circ\pi_0$ reflects into the subcategory of constant sheaves, $\shape$ reflects into the \emph{discrete} types.
Note that ``discrete'' refers to the topological structure of a type, 
not to a property of the $\infty$-groupoid structure given by its identity type.

As all modalities, $\shape$ comes with a unit-map $\sigma_X\colon X \to \shape X$, for any type $X$.
For any two points $x,y: X$ that are joined by a topological path, 
the images $\sigma_X(x)$ and $\sigma_X(y)$ are equal in $\shape X$.

\vspace{1cm}

\textbf{Contribution.} From this point on, 
let us assume that each type supposed to represent a topological space comes with a point ``$\ast$''
and for $\ast:X$ let us abbreviate $\sigma_X(\ast)$ with $\ast$.
  
For many modalities, the fibers of their units are interesting.
For $\sphere{1}$, this fiber
\[ \sum_{x:\sphere{1}}\sigma_{\sphere{1}}(x)=\ast \]
-- or more precisely its projection to $\sphere{1}$ -- turns out to be the universal cover of $\sphere{1}$.
But this works only for spaces with trivial higher homotopy groups. 
For the construction of the universal cover of an arbitrary type, this has to be adjusted:
\[ \widetilde{X}:\equiv \sum_{x\colon X} \trunc{0}{\sigma_X(x)=\ast}. \]
Note that this amounts to replacing the modality $\shape$ with the modality $\shape_1:\equiv\trunc{1}{\_}\circ\shape$.
To justify calling $\widetilde{X}$ the universal cover of $X$, 
we will define covering spaces relative to a modality and show a universal property.
    
For a modality $\bigcirc$ with unit $\eta$, we call a map $f:Y\to X$ a $\bigcirc$-cover, if the naturality square
\begin{center}
  \begin{tikzcd}
    Y\arrow[d, "f", swap]\arrow[r, "\eta_Y"] & \bigcirc Y\arrow[d, "\bigcirc f"]\drawpb{dl} \\
    X\arrow[r, "\eta_X", swap] & \bigcirc X 
  \end{tikzcd}
\end{center}
is a pullback square. 
For $\bigcirc\equiv\shape_1$, 
covering spaces in topology have the universal property of this pullback for cones with a topological space as tip.

For a $\shape_1$-cover $f:Y\to X$ there is always the trivial map $t:\widetilde{X}\to \shape_1Y$.
Together with a $g:\ast =_{\shape_1 X}\ast $ this yields a cone for the universal property of
the defining pullback of $f$ and therefore a unique map $\widetilde{X}\to Y$.

Similarly, the construction of covering spaces corresponding to subgroups of $\pi_1(X)$ 
can be done without using anything beyond the properties of a general modality:
Any subgroup $H\subseteq\pi_1(X)$ can be represented by an action of $\pi_1(X)$ on a discrete 0-type
\footnote{This means we use the \emph{homotopical} covering theory of \cite[Section 3.1]{favonia-thesis} 
and \cite[Section 7.1]{ulrik-higher-groups}}
and therefore a map $BH\to \shape_1X$, with discrete $BH$.
The pullback of this map is a $\shape_1$-cover with the correct fiber type.
A calculation on the level of abstract modalities shows, that this gives the usual one-to-one correspondence.
In its most natural abstract form this correspondence of maps $M\to \shape_1 X$ for modal $M$ 
and $\shape_1$-covers includes maps with discrete 1-types as fibers. 

For the modality $\shape$, this correspondence relates $\shape$-covers with 
maps $M\to \shape X$ for discrete $M$.
Since the latter are $\infty$-actions of $\shape X$ on discrete types, this seems to be a very natural generalization.
The universal cover construction for $\shape$ yields a ``$\shape$-universal cover'' 
which can have both non-discrete topological structure and
non-propositional identity types.
This occurs whenever $\shape X$ is not 1-truncated.
For example, if we assume a type $\CC\PP^\infty$ representing the appropriate topological space,
the $\shape$-universal cover will be a $1$-type over $\CC\PP^\infty$ with identity types merely $S^1$.
The $\shape$-universal cover $\widetilde{X}$ of a space has always a contractible shape, i.e. $\shape X \simeq 1$.

Similar generalizations of the classical topological correspondence are known on the classical side
for cohesive $\infty$-stacks \cite[Section 5.2.7]{SchreiberDcct} and topological 1-stacks \cite{noohi}.     
The author sees one advantage in the clarity of the type theoretic proofs,
since the correspondence and some related remarks are all straight forward to prove using one lemma,
which is so far the only formalized part of this work.
The statement of this lemma is a generalization of the fact that modalities preserve products:
For any dependent type $B:\bigcirc A \to \mathcal U$, we have $\bigcirc \sum_{x:A}B(\eta_A(x)) \simeq \sum_{x:\bigcirc A}\bigcirc B(x)$.


\end{document}
