\documentclass[9pt,twosided]{amsart}


\title{Discussion with Shulman, Wellen, Frey, Awodey, Rijke}

\begin{document}

\maketitle

Mike's observations:
\begin{enumerate}
\item Every factorization system $(\mathcal{E},\mathcal{M})$ has an underlying reflective
subuniverse, for which the modal types are those such that $X\to \unit$ is
in $\mathcal{M}$.  It is not clear whether every reflective subuniverse underlies
a factorization system.  One natural definition takes $\mathcal{E}$ to be the maps
inverted by the reflector $\modal$; another takes $\mathcal{M}$ to be the maps whose
$\modal$-naturality square is a pullback.  These two classes of maps are
always orthogonal, but factorizations do not always exist; i.e. they
are a "prefactorization system".  The prefactorization systems
obtained in this way from a reflective subuniverse are called
*reflective*; they are precisely those for which $\mathcal{E}$ satisfies
2-out-of-3.
\item 
\begin{enumerate}
\item Every modality is, in particular, a reflective subuniverse, so the
foregoing applies to it.  In this case, however, the putative explicit
definitions of $\mathcal{E}$ and $\mathcal{M}$ are always a factorization system, with
factorizations constructed by pulling back in the $\modal$-naturality square;
thus we always obtain a reflective factorization system.  The
modalities are precisely the reflective subuniverses that have \emph{stable
units}, i.e. all the modal units $X\to\modal X$ are $\modal$-connected.  In \cite{CassidyHebertKelly}
there are two weaker conditions on a reflective subcategory that
ensure that the reflective prefactorization system has factorizations:
being \emph{simple} (equivalent to the fact that factorizations can be
constructed by pulling back in the $\modal$-naturality square) and being
\emph{semi-left-exact}.  It is not clear whether in HoTT there exist
semi-left-exact reflective subuniverses that are not modalities, or
simple ones that are not semi-left-exact, or even reflective
subuniverses that are not simple.
\item Every modality also gives rise to a different factorization
system, in which the right class is the maps with $\modal$-modal fibers and
the left class is the maps with $\modal$-connected fibers.  This
factorization system is \emph{stable}, i.e. the left class (and hence the
factorizations) are stable under pullback.  Moreover, every stable
factorization system arises in this way from a unique modality.
\end{enumerate}
\item The above two factorization systems associated to a modality do not
in general coincide.  They can coincide only if the $\modal$-connected maps
satisfy 2-out-of-3 (since the $\modal$-inverted maps always do), and only if
the $\modal$-inverted maps are stable under pullback (since the $\modal$-connected
maps always are).  However, both of these properties are sufficient
to imply that $\modal$ is fully left exact, which in turn implies that the
two factorization systems coincide.
\item Another way of extending a reflective subuniverse to a
factorization system is that if it's accessible, we can choose a
generating family and then use the same generating family to generate
a factorization system.  This works for any accessible reflective
subuniverse, but in principle, at least, it seems as if the result may
depend on the generating family chosen, and I don't see any reason for
it to coincide with the reflective (pre)factorization system.  I also
don't know whether anything special happens for modalites.
\item If we have a factorization system $(\mathcal{E},\mathcal{M})$,
with underlying reflective subuniverse $\modal$, then all maps between
$\modal$-modal types are in $\mathcal{M}$, hence so are any of their pullbacks.  Thus,
the reflective factorization system, if it exists, is the
factorization system with the smallest class $\mathcal{M}$ that has a given
underlying reflective subuniverse.  On the other hand, again for any
$(\mathcal{E},\mathcal{M})$, any pullback of an $\mathcal{M}$-map is again in $\mathcal{M}$, hence all fibers of an
$\mathcal{M}$-map are $\modal$-modal; thus the stable factorization system, if it exists
(i.e. if $\modal$ is a modality), is the factorization system with the
\emph{largest} class $\mathcal{M}$ that has a given underlying reflective subuniverse.
\end{enumerate}

In a presheaf topos, a natural transformation is cartesian if and only if it is right orthogonal to all maps between representables. 
Question: What do we get if we localize inside an
arbitrary presheaf topos at the maps between representables?  A
general presheaf topos is not cohesive, so the discrete objects are
not fully embedded and we can't produce a modality by "taking the
colimit and then the discrete object".  But this localization seems to
be trying to give us the closest possible approximation to such.  As
an extreme example in the other direction, if the domain of presheaves
is a discrete category, there are no nontrivial maps between
representables, so every transformation is cartesian and the
localization is the identity functor.

\end{document}